\documentclass{article}
\usepackage{fullpage}
\usepackage{amsmath}
\usepackage{minted}
\usepackage{semantic}
\usepackage{xparse}
\usepackage{fancyvrb}
\usepackage{lmodern}
\usepackage{xspace}
\usepackage{mathpartir}
\usepackage{stmaryrd}
\usepackage{booktabs}

% \mathlig{|-}{\vdash}
\mathlig{<=>}{\Leftrightarrow}
\renewcommand{\k}{\kappa}
\DeclareMathSymbol{:}{\mathpunct}{operators}{"3A}
\usemintedstyle[lambda_lr_cps.py -x]{bw}
\newcommand{\own}[1]{\mathbf{own}\left(#1\right)}


\reservestyle{\keyword}{\mathbf}
\keyword{true,false,let[let\;],in[\;in\;],letcont[letcont\;],ret[\;ret\;],if[if\;],then[\;then\;],else[\;else\;],jump[jump\;],call[call\;],cont,fn,abort}
\reservestyle{\basictype}{\mathtt}
\basictype{int,bool}
\reservestyle{\auxfun}{\mathtt}
\auxfun{cgen,synth,self,sub,map,reduce}
\reservestyle{\syntaxset}{\textit}
\syntaxset{Place,Val,Op,RVal,FuncBody,BType,Type,Pred,TEnv,KEnv,Constraint}
\renewcommand{\bar}{\overline}

\newcommand{\refinement}[2]{\ensuremath{\{#1\mid#2\}}}

\RenewDocumentCommand\r
  {>{\SplitArgument{1}{|}}m}{\refinement#1 }

\newcommand{\tenv}{\mathbf{T}}
\newcommand{\kenv}{\mathbf{K}}

% \title{}
% \author{Nico Lehmann}

\begin{document}
% \maketitle

\section{Syntax}

\begin{displaymath}
  \begin{array}{rcrcl}
    \<Place>    & \ni & p     & ::=  & x \mid p.n                                                           \\
    \<Val>      & \ni & v     & ::=  & \<false> \mid \<true> \mid z                                         \\
    \<Op>       & \ni & o     & ::=  & p \mid v                                                             \\
    \<RVal>     & \ni & R     & ::=  & o\mid o_1 \oplus o_2 \mid o_1 < o_2 \mid o_1 = o_2                   \\
    \<FuncBody> & \ni & F     & ::=  & \<let> x = R \<in> F \mid \<letcont> k(\bar{x:\tau}) = F_1 \<in> F_2
    \mid \<jump> k(\bar{x})   \mid \<abort>                                                                 \\
                &     &       & \mid & \<call> f(\bar{x}) \<ret> k \mid \<if> p \<then> F_1 \<else> F_2     \\
    \<BType>    & \ni & \beta & ::=  & \<bool> \mid \<int>                                                  \\
    \<Type>     & \ni & \tau  & ::=  & \<fn>(\bar{x:\tau}) -> \tau \mid \r{x:\beta | r}
    \mid \Pi(\bar{x:\tau})                                                                                  \\
    \<Pred>     & \ni & r     & ::=  & \dots                                                                \\
    \<TEnv>     & \ni & \tenv & ::=  & \emptyset \mid \tenv, x: \tau \mid \tenv, r                          \\
    \<KEnv>     & \ni & \kenv & ::=  & \emptyset \mid \kenv, k: \<cont>(\bar{x:\tau})
  \end{array}
\end{displaymath}

\section{Typing}

\begin{mathpar}
  \textbf{Well-formed types and environments}\hfill \boxed{\tenv |- \tau \text{ and } |- \tenv}
  \\
  \infer[Wf-fn]
  {\tenv |- \Pi(\bar{x: \tau}) \\ \tenv,\bar{x:\tau} |- \tau'}
  {\tenv |- \<fn>(\bar{x:\tau}) -> \tau'}

  \infer[Wf-refine]
  {\lfloor \tenv \rfloor,x: \beta |- r \colon \<bool>}
  {\tenv |- \r{x: \beta | r}}

  \infer[Wf-prod]
  {\tenv, x_0:\tau_0,\dots,x_{i-1}: \tau_{i-1} |- \tau_{i}}
  {\tenv |- \Pi(\bar{x:\tau})}

  \infer[Wf-empty]
  {\\}
  {|- \emptyset}

  \infer[Wf-binding]
  {|- \tenv\\ \tenv |- \tau}
  {|- \tenv, x: \tau}

  \infer[Wf-pred]
  {|- \tenv\\ \tenv |- r \colon \<bool>}
  {|- \tenv, r}
\end{mathpar}

\begin{mathpar}
  \textbf{Subtyping}\hfill \boxed{\tenv |- \tau_1 \preceq \tau_2}
  \\
  \infer[$\preceq$-refine]
  {
    \mathtt{Valid}(\llbracket\tenv\rrbracket \wedge r_1 => r_2[x/y])
    \\\\
    \tenv |- \r{x: \beta | r_1}\\ \tenv |- \r{y: \beta | r_2}
  }
  {\tenv |- \r{x: \beta | r_1} \preceq \r{y: \beta | r_2}}


  \infer[$\preceq$-fun]
  {
  \tenv, \bar{y: \tau_2} |- \tau'_1[\bar{y}/\bar{x}] \preceq \tau'_2
  \\\\
  \tenv,y_0: \tau_{2_0},\dots,y_{2_{i-1}}: \tau_{2_{i-1}} |- \tau_{2_i} \preceq \tau_{1_i}[\bar{x}/\bar{y}]
  }
  {\tenv |- \<fn>(\bar{x: \tau_1}) -> \tau'_1 \preceq \<fn>(\bar{y: \tau_2}) -> \tau'_2}

  \infer[$\preceq$-prod]
  {
  \tenv,x_0: \tau_0,\dots,x_{i-1}: \tau_{i-1} |- \tau_i \preceq \tau'_i[\bar{x}/\bar{y}]
  }
  {\tenv |- \Pi(\bar{x: \tau}) \preceq \Pi(\bar{y: \tau'})}
\end{mathpar}

\begin{mathpar}
  \\
  \textbf{Well-typed rvalues}\hfill \boxed{\tenv |- R \colon T}
  \\
  \infer
  [\textsc{R-var}]
  {|- \tenv \\ \tenv(x) = \tau }
  {\tenv |- x \colon \mathtt{self}(x, \tau)}

  \infer
  [\textsc{R-proj}]
  {\tenv |- p \colon \Pi(x_1: \tau_1 \dots x_n: \tau_n \dots x_m: \tau_m)}
  {\tenv |- p.n \colon \mathtt{self}(p.n, \tau_n)}

  \infer
  [\textsc{R-add}]
  {\tenv |- o_1 \colon \r{x:\<int> | r_1} \\ \tenv |- o_2 \colon \r{x:\<int> | r_2}}
  {\tenv |- o_1 \oplus o_2 \colon \Pi(\r{x:\<int> | x = o_1 + o_2}, \<bool>)}

  \infer
  [\textsc{R-lt}]
  {\tenv |- o_1 \colon \r{x: \<int> | r_1} \\ \tenv |- o_2 \colon \r{x:\<int> | r_2}}
  {\tenv |- o_1 < o_2 \colon \r{x:\<bool> | x <=> o_1 < o_2}}

  \infer
  [\textsc{R-eq}]
  {\tenv |- o_1 \colon \r{x: \beta | r_1 }\\ \tenv |- o_2 \colon \r{x: \beta | r_2}}
  {\tenv |- o_1 = o_2 \colon \r{x:\<bool> | x <=> o_1 = o_2}}

  \infer
  [\textsc{R-bool}]
  {b=\<true> \text{ or } b=\<false>}
  {\tenv |- b \colon \r{x:\<bool> | x = b}}

  \infer
  [\textsc{R-int}]
  {\\}
  {\tenv |- z \colon \r{x:\<int> | x = z}}
\end{mathpar}

\begin{displaymath}
  \begin{array}{lcl}
    \toprule
    \mathtt{single}                                 & :: & (\<Place> \times \<Type>) -> \<Type>       \\
    \midrule
    \mathtt{single}(p, \r{x: \beta | r})            & =  & \r{x: \beta | r  \wedge x = p}             \\
    \\
    \mathtt{single}(p, \tau)                        & =  & \tau                                       \\
    \toprule
    \llbracket \cdot \rrbracket                     & :: & \<TEnv> -> \<Pred>                         \\
    \midrule
    \llbracket \emptyset \rrbracket                 & =  & \<true>                                    \\
    \\
    \llbracket \tenv,r \rrbracket                   & =  & \llbracket \tenv \rrbracket \wedge r       \\
    \\
    \llbracket \tenv,x: \r{y: \beta | r} \rrbracket & =  & \llbracket \tenv \rrbracket \wedge r[x/y]  \\
    \\
    \llbracket \tenv, \Pi(\bar{x: \tau}) \rrbracket & =  & \llbracket \tenv, \bar{x: \tau} \rrbracket \\
    \\
    \llbracket \tenv, \tau \rrbracket               & =  & \llbracket \tenv  \rrbracket               \\
    \bottomrule
  \end{array}
\end{displaymath}

\begin{mathpar}
  \\
  \textbf{Well-typed functions}\hfill \boxed{\tenv|\kenv |- F}
  \\
  \infer
  [\textsc{F-let}]
  {\tenv |- R \colon \tau \\ \tenv,x: \tau|\kenv |- F}
  {\tenv|\kenv |- \<let> x = R \<in> F}

  \infer
  [\textsc{F-letcont}]
  {
    \tenv,\bar{x:\tau}|\kenv,k: \<cont>(\bar{x: \tau}) |- F_1
    \\
    \tenv |\kenv,k: \<cont>(\bar{x:\tau}) |- F_2
    \\
    \tenv |- \Pi(\bar{x: \tau})
  }
  {\tenv|\kenv |- \<letcont> k(\bar{x:\tau}) = F_1 \<in> F_2}

  \infer
  [\textsc{F-jump}]
  {\tenv |- y_i \colon \tau_i \\ \tenv |- \tau_i \preceq \tau'_i[\bar{y}/\bar{x}]}
  {\tenv|\kenv,k: \<cont>(\bar{x:\tau'}) |- \<jump> k(\bar{y})}

  \infer
  [\textsc{F-call}]
  {
  \tenv |- y_i \colon \tau_i \\
  \tenv |- \tau_i \preceq \tau'_i[\bar{y}/\bar{x}] \\
  \tenv |- \tau_f[\bar{y}/\bar{x}] \preceq \tau_k
  }
  {\tenv,f: \<fn>(\bar{x:\tau'}) \rightarrow \tau_f \mid \kenv,k: \<cont>(z: \tau_k) |- \<call> f(\bar{y}) \<ret> k}

  \infer
  [\textsc{F-if}]
  {
    \tenv,p |\kenv |- F_1 \\
    \tenv,\neg p|\kenv |- F_2
  }
  {\tenv|\kenv |- \<if> p \<then> F_1 \<else> F_2}

  \infer
  [\textsc{F-abort}]
  {\\}
  {\tenv|\kenv |- \<abort>}
\end{mathpar}

\section{Constraint generation}
\begin{displaymath}
  \begin{array}{rcl}
    \<Constraint> \ni c & ::= & r \mid c_1 \wedge c_2 \mid \forall x: \beta. r => c \mid r => c
  \end{array}
\end{displaymath}

\begin{displaymath}
  \begin{array}{lcl}
    \toprule
    (=>)                                                                         & :: & (\<TEnv> \times \<Constraint>) -> \<Constraint> \\
    \midrule
    (\tenv, x: \r{y: \beta | r}) => c                                            & =  & \tenv => \forall x: \beta . r[x/y] => c         \\
    \\
    (\tenv, x: \Pi(\bar{y: \tau})) => c                                          & =  & (\tenv, \bar{y: \tau}) => c                     \\
    \\
    (\tenv, x: \tau) => c                                                        & =  & \tenv => c                                      \\
    \toprule
    \<sub>                                                                       & :: & (\<Type> \times \<Type>) -> \<Constraint>       \\
    \midrule
    \<sub>(\r{x: \beta | r_1}; \r{y: \beta | r_2})                               & =  & \forall x: \beta. r_1 => r_2[x/y]               \\
    \\
    \<sub>(\<fn>(\bar{x: \tau_1}) -> \tau'_1; \<fn>(\bar{y: \tau_2}) -> \tau'_2) & =  & c_0                                             \\
    \quad \mathtt{where}                                                                                                                \\
    \quad \quad c_{n+1}                                                          & =  & \<sub>(\tau'_1[\bar{y}/\bar{x}], \tau'_2)       \\
    \quad \quad c_i                                                              & =  & (y_i: \tau_{2_i}) => c_{i+1} \wedge
    \<sub>(\tau_{2_i}, \tau_{1_i}[\bar{y}/\bar{x}])                                                                                     \\
    \\
    \<sub>(\Pi(\bar{x: \tau_1}); \Pi(\bar{y: \tau_2}))                           & =  & c_0                                             \\
    \quad \mathtt{where}                                                                                                                \\
    \quad \quad c_n                                                              & =  & \<sub>(\tau_{1_n}, \tau_{2_n})                  \\
    \quad \quad c_i                                                              & =  &
    (x_i: \tau_{1_i}) => c_{i+1} \wedge \<sub>(\tau_{1_i}, \tau_{2_i})                                                                  \\
    \bottomrule
  \end{array}
\end{displaymath}


\begin{displaymath}
  \begin{array}{lcl}
    \toprule
    \<cgen>                                                            & :: & (\<TEnv> \times \<KEnv> \times \<FuncBody>) -> \<Constraint>    \\
    \midrule
    \<cgen>(\tenv; \kenv; \<let> x = R \<in> F)                        & =  & (x: \tau) => \<cgen>(\tenv,x: \tau;\kenv;F)                     \\
    \quad \mathtt{where}\quad                                                                                                                 \\
    \quad \quad\tau                                                    & =  & \<synth>(\tenv; R)                                              \\
    \\
    \<cgen>(\tenv; \kenv; \<letcont> k(\bar{x: \tau}) = F_1 \<in> F_2) & =  & (\bar{x : \tau}) => c_1 \wedge c_2                              \\
    \quad \mathtt{where}                                                                                                                      \\
    \quad \quad c_1                                                    & =  & \<cgen>(\tenv,\bar{x: \tau};\kenv,k:\<cont>(\bar{x: \tau});F_1) \\
    \quad \quad c_2                                                    & =  & \<cgen>(\tenv;\kenv,k:\<cont>(\bar{x: \tau});F_2)               \\
    \\
    \<cgen>(\tenv;\kenv;\<jump> k(\bar{y}))                            & =  & \bigwedge c_i                                                   \\
    \quad \mathtt{where}                                                                                                                      \\
    \quad \quad \<cont>(\bar{x:\tau'})                                 & =  & \kenv(k)                                                        \\
    \quad \quad \tau_i                                                 & =  & \<synth>(\tenv; y_i)                                            \\
    \quad \quad c_i                                                    & =  & \<sub>(\tau_i, \tau'_i[\bar{y}/\bar{x}])                        \\
    \\
    \<cgen>(\tenv;\kenv; \<call> f(\bar{y}) \<ret> k)                  & =  & \bigwedge c_i \wedge c                                          \\
    \quad \mathtt{where}                                                                                                                      \\
    \quad \quad \<fn>(\bar{x: \tau'})-> \tau_f                         & =  & \tenv(f)                                                        \\
    \quad \quad \<cont>(z: \tau_k)                                     & =  & \kenv(k)                                                        \\
    \quad \quad \tau_i                                                 & =  & \<synth>(\tenv; y)                                              \\
    \quad \quad c_i                                                    & =  & \<sub>(\tau_i, \tau'_i[\bar{y}/\bar{x}])                        \\
    \quad \quad c                                                      & =  & \<sub>(\tau_f[\bar{y}/\bar{x}], \tau_k)                         \\
    \\
    \<cgen>(\tenv;\kenv;\<if> p \<then> F_1 \<else> F_2)               & =  & c_1 \wedge c_2                                                  \\
    \quad \mathtt{where}                                                                                                                      \\
    \quad \quad c_1                                                    & =  & p => \<cgen>(\tenv;\kenv;F_1)                                   \\
    \quad \quad c_2                                                    & =  & \neg p => \<cgen>(\tenv;\kenv;F_2)                              \\
    \\
    \<cgen>(\tenv;\kenv;\<abort>)                                      & =  & \<true>                                                         \\
    \toprule
    \<synth>                                                           & :: & (\<TEnv> \times \<RVal>) -> Type                                \\
    \midrule
    \<synth>(\tenv;x)                                                  & =  & \<self>(x; \tenv(x))                                            \\
    \<synth>(\tenv;p_1 \oplus p_2)                                     & =  & \Pi(\r{x: \<int> | p_1 + p_2}, \<bool>)                         \\
    \dots                                                                                                                                     \\
    \bottomrule
  \end{array}
\end{displaymath}

\newpage

\section{Examples}

\subsection{Example 1}

\begin{minted}{rust}
fn sum(n: {i32 | n >= 0}) -> {v: i32 | v >= n} {
  let mut i = 0;
  let mut r = 0;
  while (i <= n) {
    r += i;
    i += 1;
  }
  r
}
\end{minted}

\begin{minted}[escapeinside=\%\%,fontfamily=courier, mathescape=true]{lambda_lr_cps.py -x}
fn sum(n: {i32 | n %$\ge$% 0}) ret k(v: {i32 | v %$\ge$% n}) =
  letcont loop(i1: {i32 | i1 %$\ge$% 0}, r1: {i32 | r1 %$\ge$% i1)} =
    let t0 = i1 %$\le$% n in
    if t0 then
      let t1 = r1 + i1 in
      if t1.1 then
        let t2 = i1 + 1 in
        if t2.1 then jump loop(t1.0, t2.0) else abort
      else
        abort
    else
      jump k(r2)
  in
  let i0 = 0, r0 = 0 in jump loop(i0, r0)
\end{minted}

\newpage
\subsection{Example 2}

\begin{minted}{rust}
fn f(n: {i32 | n >= 0}) -> i32;

fn count_zeros(limit: {i32 | n >= 0}) -> {v: i32 | v >= 0} {
  let mut i = 0;
  let mut c = 0;
  while (i < limit) {
    if (f(i) == 0) {
      c += 1;
    }
    i += 1;
  }
  c
}
\end{minted}

\begin{minted}[escapeinside=\%\%,fontfamily=courier, mathescape=true]{lambda_lr_cps.py -x}
fn count_zeros(limit: {i32 | n %$\geq$% 0}) ret k(v: {i32 | v >= 0}) =
  letcont b0(i1: {i32 | n %$\geq$% 0}, c1: {i32 | c %$\geq$% 0}) =
    let t0 = i1 < limit in
    if t0 then
      letcont b1(x: i32) =
        letcont b2(c3: {i32 | c3 %$\geq$% 0}) =
          let t3 = i1 + 1 in
          if t3.1 then jump b0(i2, t3.0) else abort
        in
        let t1 = x == 0 in
        if t1 in
          let t2 = c1 + 1 in
          if t2.1 then jump b2(t2.0) else abort
        else
          jump b2(c1)
      in
      call f(i1) ret b1
    else
      jump k(c1)
  in
  let i0 = 0, c0 = 0 in jump b0(i0, c0)
\end{minted}

\newpage
\section{Examples with references}

\subsection{Tracking variable versions}
\begin{minted}{rust}
fn ris() {
  let mut p = (1, 2);
  let x = p.0;
  p.0 = 3;
  let y = p.0;
  let z = p.1;
  // What can we say about x, y and z?
}
\end{minted}

\begin{minted}[escapeinside=\%\%,fontfamily=courier,mathescape]{lambda_lr_cps.py -x}
fn ris() ret k(()) =
  let p = new(2) in   // $\ell^p_0: \lightning_2, p: \own{\ell^p_0}$
  p.0 := 1;           // $\dots, \ell^p_1: \r{i32 | \nu = 1} \times \lightning_1, p: \own{\ell^p_1}$
  p.1 := 2;           // $\dots, \ell^p_2: \r{i32 | \nu = 1} \times \r{i32 | \nu = 2}, p: \own{\ell^p_2}$
  let x = new(1) in   // $\dots, \ell^x_0: \lightning_1, x: \own{\ell^x_0}$
  x := *p.0;          // $\dots, \ell^x_1: \r{i32 | \nu = \ell^p_2.0}, x: \own{\ell^x_1}$
  p.0 := 3;           // $\dots, \ell^p_3: \r{i32 | \nu = 3} \times \r{i32 | \nu = 2}, p: \own{\ell^p_3}$
  let y = new(1) in   // $\dots, \ell^y_0: \lightning_1, y: \own{\ell^y_0}$
  y := *p.0;          // $\dots, \ell^y_0: \r{i32 | \nu = \ell^p_3.0}, y: \own{\ell^y_1}$
  let z = new(1) in   // $\dots, \ell^z_0: \lightning_1, y: \own{\ell^z_0}$
  z := *p.1;          // $\dots, \ell^z_1: \r{i32 | \nu = \ell^p_3.1}, z: \own{\ell^z_1}$
  jump(())
\end{minted}

\subsection{Basic control flow}

\begin{minted}{rust}
fn abs(mut x: i32) -> i32 {
  if (x < 0)  {
    x -= 1;
  }
  x
}
\end{minted}

\begin{minted}[escapeinside=\%\%,fontfamily=courier,mathescape]{lambda_lr_cps.py -x}
fn abs(x: own(i32)) ret k(own({i32 | %$\nu$% > 0})) =
  // $\ell^x_0: i32, x: \own{\ell^x_0}$
  let t0 = *x in     // $\dots, t_0: \r{i32 | \nu = \ell^x_0}$
  let b = t0 < 0 in  // $\dots, b: \r{bool | \nu \Leftrightarrow 0 > t_0}$
  if b then          // $\dots, b$
    let t1 = *x in   // $\dots, t_1: \r{i32 | \nu = \ell^x_0}$
    x := t1 - 1      // $\dots, \ell^x_1: \r{i32 | \nu = t_1 \text{ - } 1}, x: \own{\ell^x_1}$
    jump k(x)
  else               // $\dots, \neg b$
    jump k(x)
\end{minted}


\clearpage
\subsection{Dependent function}
\begin{minted}[escapeinside=\%\%]{rust}
fn ira(a: i32, b: {i32 | b > a}) -> {i32 | v > 0} {
  b - a
}
\end{minted}

\begin{minted}[escapeinside=\%\%,fontfamily=courier,mathescape]{lambda_lr_cps.py -x}
fn ira(a: own(%$\ell^a_0$%, i32), b: own(%$\ell^b_0$%, {i32 | %$\nu$% > %$\ell^a_0$%})) ret k(own({i32 | %$\nu$% > 0})) =
  // $\ell^a_0: i32, a: \own{\ell^a_0}, \ell^b_0: \r{i32 | \nu > \ell^a_0}, b: \own{\ell^b_0}$
  let t0 = new(1) in   // $\dots, \ell^{t_0}_0: \lightning_1, t_0: \own{\ell^{t_0}_0}$
  let t1 = *a in       // $\dots, t_1: \r{i32 | \nu = \ell^a_0}$
  let t2 = *b in       // $\dots, t_2: \r{i32 | \nu = \ell^b_0}$
  let t3 = t2 - t1 in  // $\dots, t_3: \r{i32 | \nu = t_2 \text{ - } t_1}$
  t0 := t3;            // $\dots, \ell^{t_0}_1: \r{i32 | \nu = t_3}, t_0: \own{\ell^{t_0}_1}$
  jump k(t0)
\end{minted}

\subsection{The sum example with references}
\begin{minted}[escapeinside=\%\%]{rust}
fn sum(n: i32) -> {i32 | v >= n} {
  let mut i = 0;
  let mut r = 0;
  while (i < n) {
    i += 1;
    r += i;
  }
  r
}
\end{minted}

\begin{minted}[escapeinside=\%\%,fontfamily=courier,mathescape]{lambda_lr_cps.py -x}
fn sum(n: own(%$\ell^n_0$%, i32)) ret k(own({i32 | %$\nu$% %$\geq$% %$\ell^n_0$%})) =
  // $\ell^n_0: i32, n: \own{\ell^n_0}$
  letcont loop(i: own(%$\ell^i_1$%, i32), r: own(%$\ell^r_1$%, {i32 | %$\nu$% %$\geq$% %$\ell^i_1$%})) =
    // $\dots, \ell^i_1: i32, i: \own{\ell^i_1}, \ell^r_1: \r{i32 | \nu \geq \ell^i_1}, r: \own{\ell^r_1}$
    let t0 = *i in     // $\dots, t_0: \r{i32 | \nu = \ell^i_0}$
    let t1 = *n in     // $\dots, t_1: \r{i32 | \nu = \ell^n_0}$
    let b = t0 < t1 in // $\dots, b: \r{i32 | \nu \Leftrightarrow t_1 > t_0}$
    if b then          // $\dots, b$
      i := t0 + 1;     // $\dots, \ell^i_2: \r{i32 | \nu = t_0 + 1}, i: \own{\ell^i_2}$
      let t2 = *i in   // $\dots, t_2: \r{i32 | \nu = \ell^i_2}$
      let t3 = *r in   // $\dots, t_3: \r{i32 | \nu = \ell^r_2}$
      r := t2 + t3;    // $\dots, \ell^r_2: \r{i32 | \nu = t_2 + t_3}, r: \own{\ell^r_2} $
      jump loop(i, r)
    else
      jump k(r)
  in
  let i = new(1) in
  i := 0;            // $\dots, \ell^i_0: \r{i32 | \nu = 0}, i: \own{\ell^i_0}$
  let r = new(1) in
  r := 0;            // $\dots, \ell^r_0: \r{i32 | \nu = 0}, r: \own{\ell^r_0}$
  jump loop(i, r)
\end{minted}

\clearpage
\subsection{Non-copy type}
\begin{minted}{rust}
// Types are by default non-copy
struct Point {
  x: i32,
  y: i32
}

fn hipa(r: (Point, Point)) -> i32 {
  let a = r.0;
  a.x + r.1.y
}
\end{minted}

\begin{minted}[escapeinside=\%\%,fontfamily=courier,mathescape]{lambda_lr_cps.py -x}
fn hipa(r: own%$(\ell^{r}_0$%, Point %$\times$% Point)) ret k(own(i32)) =
  // $\ell^{r}_0: Point \times Point, r: \own{\ell^r_0}$
  let a = new(Point) in
  a := *r.0;             // $\ell^r_0.0: Point, \ell^r_0.1: Point, r.1: \own{\ell^r_0.1}, a: \own{\ell^r_0.1}$
  let t0 = *a.0 in       // $\dots, t_0: \r{i32 | \nu = \ell^r_0.0}$
  let t1 = *r.1.y in     // $\dots, t_1: \r{i32 | \nu = \ell^r_0.1}$
  let t2 = t0 + t1 in    // $\dots, t_2: \r{i32 | \nu = t_0 + t_1}$
  jump k(t2)
\end{minted}
\end{document}
